\section{Challenges and Significance}

Financial markets are inherently complex, characterized by:

\begin{itemize}
\item \textbf{Noise:} The presence of random fluctuations and irrelevant information.
\item \textbf{Non-stationarity:} The underlying statistical properties of the data change over time.
\item \textbf{Unpredictable external factors:} News events, economic shifts, and geopolitical tensions can significantly impact market behavior.
\end{itemize}

Our model will initially assume no slippage in the backtesting phase. However, to ensure realistic performance evaluation, we will incorporate slippage-adjusted simulations and liquidity constraints. This will help to mitigate execution risks and provide a more accurate assessment of the strategy's real-world viability.

\section{Data}

We will collect the following data on the constituents of the S\&P 500 from yfinance, covering the past \textit{n} years:

\begin{itemize}
\item \textbf{Price Data:} Open, High, Low, Close, Adjusted Close, Volume.
\item \textbf{Fundamental Data:} Market Cap, Earning Reports, PE Ratios.
\end{itemize}

\section{Approach}

We will define the prediction problem as a classification task for our LSTM network. Our primary reference material will be "Deep learning with long short-term memory networks for financial market predictions" by Thomas Fischer and Christopher Krauss. We will prioritize the functionality of our model and trading strategy.

\section{Evaluation}

The project's success will be evaluated based on:

\begin{itemize}
\item \textbf{Backtesting:} The performance of the strategy will be assessed through backtesting on historical S\&P 500 data.
\item \textbf{Functionality:} The primary focus will be on developing a functional LSTM network and trading strategy.
\end{itemize}

\end{document}
